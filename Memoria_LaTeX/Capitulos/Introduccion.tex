\chapter{Introducción}
\label{cap:introduccion}

%\chapterquote{Frase célebre dicha por alguien inteligente}{Autor}

% Según la normativa  para Trabajos de Fin de Grado\footnote{\url{https://informatica.ucm.es/file/normativatfg-2021-2022?ver} (ver versión actualizada para cada curso académico)}, la memoria incluirá una portada normalizada con la siguiente información: título en castellano, título en inglés, autores, profesor director, codirector si es el caso, curso
% académico e identificación de la asignatura (Trabajo de fin de grado del Grado en -
% nombre del grado correspondiente-, Facultad de Informática, Universidad Complutense de Madrid). Los datos referentes al título y director (y codirector en su caso) deben corresponder a los publicados en la página web de TFG.

% La memoria debe incluir la descripción detallada de la propuesta hardware/software realizada y ha de contener:
% \begin{enumerate}
% \renewcommand{\theenumi}{\alph{enumi}}
% 	\item un índice,
% 	\item un resumen y una lista de no más de 10 palabras clave para su búsqueda bibliográfica, ambos en castellano e inglés,
% 	\item una introducción con los antecedentes, objetivos y plan de trabajo,
% 	\item resultados y discusión crítica y razonada de los mismos, con sus conclusiones,
% 	\item bibliografía.
% \end{enumerate}
% Para facilitar la escritura de la memoria siguiendo esta estructura, el estudiante podrá usar las plantillas en LaTeX o Word preparadas al efecto y publicadas en la página web de TFG.

% La memoria constará de un mínimo de 25 páginas para los proyectos realizados por un único estudiante, y de al menos 5 páginas más por cada integrante adicional del grupo. En este número de páginas solo se tiene en cuenta el contenido correspondiente a los apartados c y d del punto anterior.

% La memoria puede estar escrita en castellano o inglés, pero en el primer caso la introducción y las conclusiones deben aparecer también en inglés. Las memorias de los TFG matriculados en el grupo I deberán estar escritas íntegramente en inglés, excepto por lo especificado en los puntos 1 y 2 anteriores (título, resumen y lista de palabras clave).

% En caso de trabajos no unipersonales, cada participante indicará en la memoria su contribución al proyecto con una extensión de al menos dos páginas por cada uno de los participantes.

% Todo el material no original, ya sea texto o figuras, deberá ser convenientemente citado y referenciado. En el caso de material complementario se deben respetar las licencias y \emph{copyrights} asociados al software y hardware que se emplee. En caso contrario no se autorizará la defensa, sin menoscabo de otras acciones que correspondan.

Con el paso de los años, los videojuegos se han convertido en una de las industrias de entretenimiento más grandes del mundo, y con este crecimiento viene la necesidad de adaptarse a diferentes lenguajes y culturas.
En este punto es donde entran en acción los procesos de \textbf{internacionalización} (\textbf{\textit{I18N}}) y \textbf{localización} (\textbf{\textit{L10N}}).
La \textbf{internacionalización} es el proceso por el cual un videojuego se prepara desde sus etapas iniciales de desarrollo para soportar diferentes \textbf{culturas} e \textbf{idiomas}, de tal manera que no haya que realizar grandes modificaciones en el \textbf{código} para introducirlos.
Por otra parte, la \textbf{localización} es el proceso por el cual se \textbf{traducen} y adaptan los textos, gráficos y recursos de un videojuego para las diferentes culturas en las que éste se va a comercializar.

Durante ambos de estos procesos se pueden producir distintos tipos de errores (como errores de traducción o errores de visualización), aquí es donde entra el \textbf{\textit{Localization Quality Assurance}} (o \textbf{LQA}). Este proceso se encarga de revisar y probar las distintas partes del videojuego para asegurarse de que no se produzca ninguno de estos errores y el producto final sea adecuado para las culturas y el público objetivo.

Hasta ahora este proceso ha sido un trabajo manual en el que los probadores comprueban meticulosamente cada texto y como éstos se visualizan dentro del videojuego, lo cual requiere de una gran cantidad de tiempo y recursos. Aunque existen herramientas para comprobar la corrección de las traducciones y los textos automáticamente, sin embargo, no se puede decir lo mismo en lo referente a comprobar que los textos se visualicen correctamente en el contexto del videojuego.

\section{Motivación}
Teniendo en cuenta lo expuesto en la sección anterior, nuestra motivación para este trabajo es poder automatizar las pruebas sobre los procesos de internacionalización y localización, de forma que se puedan
ahorrar tiempo y recursos en estos campos y dedicarlos a otros más importantes. Para conseguir esto, vamos a utilizar técnicas de visión por computador y Reconocimiento óptico de caracteres (OCR, por sus siglas en inglés)
de forma que a partir de una captura de pantalla del videojuego, se pueda extraer el texto y realizar distintas comprobaciones y pruebas sobre él.


\section{Objetivos}
Nuestros objetivos en este trabajo son los siguientes:
\begin{enumerate}
	\item Utilizar tecnología de OCR para reconocer el texto en capturas de pantalla de un videojuego.
	\item Crear una \textit{suite} de \textit{testing} que sea capaz de realizar pruebas para errores de internacionalización y localización.
	\item Combinar estas dos partes para utilizar el texto reconocido en las imágenes como entrada de estas pruebas.
\end{enumerate}


\section{Plan de trabajo}
Para conseguir nuestros objetimos, seguiremos los siguientes pasos:
\begin{itemize}
	\item Ivestigar sobre el proceso y las herramientas utilizadas en LQA, además de los errores más comunes en internacionalización y localización.
	\item Investigar sobre OCRs y cómo utilizarlas, para elegir una adecuada a nuestras especificaciones, realizando pruebas y comparaciones entre ellas.
	\item Crear una \textit{suite} de \textit{testing}, que distribuimos en una imagen de Docker para que pueda desplegarse en cualquier equipo.
	\item Evaluar nuestras pruebas para asegurarnos de que tienen un procentaje de acierto adecuado.
\end{itemize}


% \section{Explicaciones adicionales sobre el uso de esta plantilla}
% Si quieres cambiar el \textbf{estilo del título} de los capítulos del documento, edita el fichero \verb|TeXiS\TeXiS_pream.tex| y comenta la línea \verb|\usepackage[Lenny]{fncychap}| para dejar el estilo básico de \LaTeX.

% Si no te gusta que no haya \textbf{espacios entre párrafos} y quieres dejar un pequeño espacio en blanco, no metas saltos de línea (\verb|\\|) al final de los párrafos. En su lugar, busca el comando  \verb|\setlength{\parskip}{0.2ex}| en \verb|TeXiS\TeXiS_pream.tex| y aumenta el valor de $0.2ex$ a, por ejemplo, $1ex$.

% TFGTeXiS se ha elaborado a partir de la plantilla de TeXiS\footnote{\url{http://gaia.fdi.ucm.es/research/texis/}}, creada por Marco Antonio y Pedro Pablo Gómez Martín para escribir su tesis doctoral. Para explicaciones más extensas y detalladas sobre cómo usar esta plantilla, recomendamos la lectura del documento \texttt{TeXiS-Manual-1.0.pdf} que acompaña a esta plantilla.

% El siguiente texto se genera con el comando \verb|\lipsum[2-20]| que viene a continuación en el fichero .tex. El único propósito es mostrar el aspecto de las páginas usando esta plantilla. Quita este comando y, si quieres, comenta o elimina el paquete \textit{lipsum} al final de \verb|TeXiS\TeXiS_pream.tex|

% \subsection{Texto de prueba}
