\chapter*{Resumen}

\section*{\tituloPortadaVal}

La localización es una tarea imprescindible de los videojuegos hoy en día, ya que los videojuegos aspiran a venderse y publicarse en una gran variedad de países. 
Para ello, es necesario adaptar los videojuegos a diferentes lenguajes y culturas. A ese proceso lo llamamos localización. El trabajo del equipo de localización puede conllevar gran cantidad de iteraciones para evitar errores que puedan aparecer.

Hoy en día no existen programas que verifiquen si un videojuego tiene errores de localización, ya sean de traducción o de internacionalización, por lo que es necesario personal que tenga que hacer esa tarea de verificación, esto supone un alto coste en tiempo y dinero.

El objetivo de este trabajo es tratar de automatizar esas tareas, empezando por reconocer y recoger el texto que aparece en el videojuego desde una imagen, seguido de unas pruebas que verifican si el texto reconocido tiene algún error de localización, generando un documento que los indique.


\section*{Palabras clave}
   
\noindent Videojuegos, Localización, Internacionalización, Automatización, QA

   


